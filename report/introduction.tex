\section{Introduction}
This technical report presents a general pipeline to analyse the features that can be used to predict transients faults during execution of a program. Transient faults are also known as soft errors or single event upsets. These faults are caused by either alpha particles stemming from radioactive decay, or neutrons that are present in the atmosphere. These external particles add or remove charge causing electron or hole pairs absorbed by source or drain diffusion process. This fault happens for a very short period of time and hence can be masked at various stages due to various aspects. Few of the masking factors are a) logic masking b) timing masking, c) electrical masking, d) microarchitectural masking and e) software masking. Due to the presence of so many masking levels the transient faults are 80\% of the time masked. However, there are 20\% chance that a fault can show up as a fault (segmentation) in the execution of an application or given an incorrect output.

This project is motivated to find these unmasked transient faults during the execution of a program in order to provide an ability to stop the execution if a fault as occurred, flush the pipeline and run the program from the beginning as shown in the Figure~\ref{fig:teaser}. These faults are often analysed using microarchitecture simulators. Following this practice, we utilize GemFI \cite{parasyris2014gemfi} a microarchitecture simulator extended based on Gem5 \cite{Binkert:2011:GS:2024716.2024718} to provide a fault injection tool. Using GemFI we study the fault injection mechanism and how it affects applications. In addition to that we analyse the microarchitectural events captured by the simulator to know the status of the execution. These events can be used as a features to predict the occurance of the transient faults.

Our contribution is two-fold: 1) we implemented a general pipeline that can extract meaningful features from architectural events for any target application on GemFI simulator, 2) we provide analysis on our prediction mechanism using RandomForest algorithm along with our observations.

