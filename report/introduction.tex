\section{Introduction}
This technical report presents a general pipeline to analyze the features that can predict transients faults during execution of a program. These faults are caused by either alpha particles stemming from radioactive decay, or neutrons that are present in the atmosphere. These external particles add or remove charge causing electron or hole pairs absorbed by source or drain diffusion process. These faults occur for a very short period of time and gets masked at various stages execution. Some of the factors responsible for this masking are a) logic masking b) timing masking, c) electrical masking, d) microarchitectural masking and e) software masking. Due to the presence of so many masking factors, the transient faults are masked 80\% of the time. However, the 20\% chance to produce a fault prevents it from being used in extreme conditions. These faults generally results in a fault (segmentation) in the execution of an application or produce an incorrect output.

This project is motivated to find these unmasked transient faults during the execution of a program in order to provide an ability to stop the execution if a fault as occurred, flush the pipeline and run the program from the beginning as shown in the Figure~\ref{fig:teaser}. The faults are often analysed using microarchitecture simulators. Following this practice, we utilize GemFI \cite{parasyris2014gemfi} a microarchitectural simulator which is extended based on Gem5 \cite{Binkert:2011:GS:2024716.2024718} to provide a fault injection tool. Using GemFI we study the fault injection mechanism and how it affects applications. In addition to that we analyse the microarchitectural events captured by the simulator to know the status of the execution. In this work we claim that the architectural events can be used as a features to predict the occurrence of the transient faults.

Our contribution is two-fold: 1) we implemented a general pipeline that can extract meaningful features from architectural events for any target application on GemFI simulator\footnote{https://github.com/zqsui/EECS573\_finalproj}, 2) we provide analysis on our prediction mechanism using RandomForest \cite{breiman2001random} algorithm along with our observations.

